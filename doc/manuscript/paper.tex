%\documentclass[2columns]{article}

\documentclass[letterpaper,12pt]{article}
\usepackage{url}
\usepackage{graphicx}
\usepackage{url}
\usepackage{wrapfig}
\usepackage[normalem]{ulem}
\usepackage{float}
\usepackage{xr}

\usepackage{amsmath,amssymb,amsfonts,amscd,latexsym}
\usepackage{tabularx}
\usepackage{multirow}
\usepackage{multicol}
\usepackage{eurosym}
\usepackage{geometry}
\usepackage{fancyhdr}
\usepackage{enumitem}
\usepackage{xspace}

\usepackage{booktabs}
\usepackage[svgnames,table]{xcolor}
\usepackage[tableposition=above]{caption}
\usepackage{pifont}

\newcommand*\CHECK{\ding{51}}

% Two column foonotes
\usepackage{dblfnote}


%% This is the recommended preamble for your document.

%% Load De Gruyter specific settings 
\usepackage{dgjournal}          

%% The mathptmx package is recommended for Times compatible math symbols.
%% Use mtpro2 or mathtime instead of mathptmx if you have the commercially
%% available MathTime fonts.
%% Other options are txfonts (free) or belleek (free) or TM-Math (commercial)
\usepackage{mathptmx}

%% Use the graphics package to include figures
\usepackage{graphics}

%% Use natbib with these recommended options
\usepackage[authoryear,comma,longnamesfirst,sectionbib]{natbib} 


%\externaldocument{supp}


\newcommand{\todo}[1]{\textbf{[TODO: #1]}}
\newcommand{\note}[1]{\textbf{[NOTE: #1]}}
\newcommand{\fixme}[1]{\textbf{[FIXME: #1]}}

\newcommand{\OMIT}[1]{}
\newcommand{\Xb}{\textbf{X}}
\newcommand{\RR}{\mathbb{R}}
\newcommand{\Dcal}{\mathcal{D}}
\newcommand {\br}[1]{\left(#1\right)}

\begin{document}

% Title of paper
\title{Analysing the three-dimensional structure of {\em P. falciparum}}
\author{Nelle Varoquaux}

%\maketitle


\begin{abstract}

\end{abstract}


\section{Introduction}
\label{sec:introduction}

With more than 217 millions cases nearly 429 000 deaths in 2015,  malaria
remains a major disease burden in tropical and subtropical countries and an
impediment to economic development. In Africa, where more than 90\% of the
cases and deaths occurs, the effects of malaria extend far beyond direct
measures of mortality and the disease is thought to cost more than US\$21
billion a year \citep{onwejkwe:do}.

Malaria is caused by small unicellular protozoan {\em Plasmodium}, infecting
its host through a mosquito bite. Out of the 5 species that can infect human,
{\em P. falciparum} is by the far the most deadly with 99\% of the deaths
associated to it. The parasite has a complex life cycle, with multiple stages
both in the human and mosquito host. The human host is infected with
\textit{sporozoites} through the bite of a infected female  {\em Anopheles}
mosquito. The \textit{sporozoites} quickly migrate to the liver, where they
start a two weeks multiplication process. They are then released as
\textit{merozoites} in the bloodstream and proceed to infect red blood cells. The
parasites then start their ``erythrocytic'' cycle (through \textit{rings},
\textit{trophozoites} and \textit{schizonts} stages), via another round of
replications in red blood cells, through an unusual process of cell division
called \textit{schizogony}. The parasite first undergoes multiple rounds of nuclear
replication and division into 13 to 32 daughter cells, until the red blood
cell bursts and provokes the release of \textit{merezoites} and the infective cycle
starts anew. This asexual replication cycle is responsible for the symptoms
and the complications of the disease: anemia, tertian fever, \dots

While efficacious vaccines still remains a hope and drug resistance to anti
malaria continues to arise, genomic-based research on malaria seeks new
avenues for developing novel therapies \citet{kirchner:recent}. One of the
limiting factors of developments of new drugs is our poor understanding of the
mechanisms underlying the parasite's complex life cycle. While the development
of {\em P. falciparum} through the different stages of its life is thought to
be driven by coordinated changes in gene expression, the relative paucity of
transcription factors points to unusual gene regulatory mechanisms . Meanwhile,
the relative abundance of proteins related to chromatin structures, mRNA decay
and translation rates suggests alternative mechanisms of gene regulation at the
epigenetic and post-translational levels
\citep{cui:chromatin-mediated, duffy:role, hoeijmakers:placing,
horrocks:control, deitsch:mechanisms}.

In recent year, chromosome conformation captures-like method, broadly referred
to as Hi-C allowed to identify physical interactions between two regions in a
genome-wide fashion, yielding information on their relative spatial distance
in the nucleus. Hi-C has opened new avenues for more systematic analysis of
the three-dimensional folding of the genome, paving the way for a better
understanding of the relations between 3D structure and gene regulation,
replication timing, epigenetic changes as well as many other biological
processes. The 3D genome of a wide variety of organisms has been studied in
the past year, including many species of yeasts \citet{duan:three-dimensional,
burton:xx}, bacteria, flies, plants and numerous human and mouse cell lines.
Within the past few years, two studies of the three-dimensional structure of
{\em P. falciparum} have been released.  \citet{lemieux:genome-wide} probed
several strains of {\em P. falciparum} to understand the link between 3D
structure and the complex regulation of \textit{var} genes, a family of genes
involved the invasion of red blood cell and also responsible for the
parasite's great capacity to evade immuno-therapy.
\citet{ay:three-dimensional} assayed the 3D structure of the genome of the
parasite during three key stages of the erythrocytic cycle.
\citet{ay:multiple} reviews the key biological findings of those studies.

An important part of analysing the three-dimensional structure of the
genome of {\em P. falciparum} (as well as many other organisms) relied on the
inferring accurate 3D models of the structure. These models were then used as
a preliminary step to various analysis, such as identifying colocolized
elements, distinguishing between open and closed chromatin and last but not
least, intergrating with other sources of data, such as gene expression or
chromatin modification. 

The study of chromosome organization based on contact count maps broadly
falls into two categories: model-based studies and data-driven studies. The
former methods consider the polymer nature of DNA to leverage the theoretical
and computational work done in statistical physics of polymers to build with
as few assumptions as possible many chromosome conformations. Those chromosome
conformations are then used to compare against experimental data, such as Hi-C
contact count matrices, in order to iteratively improve the models. These
models offer mechanistical insights into the folding of DNA. The latter
approaches use the experimental data to infer 3D models, by typically minizing
a cost function ensuring the models are as consistent as possible with the
data. These data driven models and analysis, with a particular focus on the
one applied to {\em P. falciparum} are the primary focus of this
review.

% XXX See rosa:computational

Though we here review some of the methods used to study and build models, this
is a very incomplete view of a blooming field. \citet{rosa:computational}
provide a more thorough (but again incomplete) overview of computational models
of genome architectures.


\section*{Three-dimensional models of DNA from Hi-C data}

\todo{A revoir}

In this section, we review ``data-driven'' methods to infer 3D models of DNA
from Hi-C data. These methods broadly fall into two categories: (i)
\textit{consensus} approaches, that aim at inferring a unique mean structures
best representing the contact count data; (ii) \textit{ensemble} methods that
yield a population of structures.

Both consensus and ensemble methods have benefits and drawbacks. Ensemble
approaches are more biologically accurate: Hi-C data are derived from a
population of cells, each of those with a uniquely folded 3D structure. The
variability of structures among ours cells may thus be better represented by a
population of structures. Yet, in addition of being more complex and
computationally intensive, ensemble approaches raise the question of
interpretability. Often, one has to fall back to interpreting the mean
structure \citep{kalhor:genome}, or a reduced set of structures
\citep{rousseau:three}. On the other hand, consensus approaches yield a single
structure recapitulating the rich information provided by Hi-C data, including
hallmarks of genome architecture shared across all cells despite the
cell-to-cell variability \citet{nagano:single-cell}. More amenable for
visualization and analysis, this average structure can be easily integrated
with other sources of data, such as RNA-seq, chip-seq etc, which are also
population-based.


After introducing some notations and common terminology, I will dive both in
consensus and population inference methods.

\subsection*{Notation}

A typical 3C/Hi-C experiment can be summarized as a contact
count matrix $\mathbf{C} \in \RR^{n \times n}$, where each row and column
correspond to a genomic-window (or loci), and each entry $c_{ij}$ to the
number of times loci $i$ and $j$ have been observed interacting. Contact count
matrices are subjected to biological and technical biases that need to be
corrected \citep{imakaev:iterative, cournac:normalization, yaffe:probabilistic,
hu:hicnorm}.

Methods model chromosomes as a series of beads in 3D, each bead corresponding
to a specific genomic window. We denote by $\textbf{X} \in \RR^{3 \times n}$
the coordinate matrix of the structure(s), $x_i \in \RR^3$ corresponds to
the coordinate matrix of the i-th beads and $d_{ij} = \|x_i - x_j\|_2$ the
euclidean distance separating beads $i$ and $j$.

\subsection*{Consensus models}

Consensus methods aim at inferring a model that best represents the contact
count matrix. They model each chromosome by a chain of beads, and attempt to
place the beads in a 3D space to minimize an objective function
$\mathcal{O}(\mathbf{O}, \mathbf{C})$, sometimes
under constraints.

\begin{equation*}
\renewcommand{\arraystretch}{2}
\begin{array}{ccl}
\underset{\Xb}{\text{minimize}} & & \mathcal{O}(\mathbf{C}, \Xb)
\end{array}
\end{equation*}

\subsubsection*{Metric MDS-based methods}

Early methods \citep{dekker:capturing, duan:three-dimensional,
tanizawa:mapping, ay:three-dimensional} cast the 3D inference problem as a
multidimensional scaling (MDS) problem: chromosomes are models as a chain of
beads, to place in 3D such that the distance between bead $i$ and $j$ matches
as closely as possible a ``wish-distance'' $\delta_{ij}$ derived from contact
counts.

\begin{equation*}
\renewcommand{\arraystretch}{2}
\begin{array}{ccl}
\underset{\Xb}{\text{minimize}} & & \underset{i, j \in \mathcal{D}}{\sum}
\frac{1}{w_{ij}} (d_{ij} - \delta_{ij})^2 \\
\text{subject to} & & \text{Biologically motivated constraints} \\
\end{array}
\end{equation*}


The first step is thus to find an adequate count-to-distance mapping to
convert contact counts into pairwise wish-distances $\delta_{ij}$.
\citet{dekker:capturing} models the 78 pairwise contact count of {\em S.
cerevisiae}'s chromosome III as a Worm-like chain polymer,
\citet{duan:three-dimensional} uses a linear mapping between contact counts
and wish distances, \citet{ay:three-dimensional} relies on the biophysical
properties of fractal globule polymers, \citet{tanizawa:mapping} inferred the
count-to-distance mapping by fitting the relationship using known pairwise
distances obtained through high-resolution FISH measures. While parameters of
these counts-to-distance functions vary, most of them can be summarized as a
power-law relationship: $c_{ij} = \beta d_{ij}^\alpha$.

\citet{lesne:3d} proposes a different approach to convert counts into
distances: they borrow the concept of shortest path from graph theory to
create a distance matrix. A graph is constructed from the contact map by
considering each loci as a node. Loci seen interacting are connected through
an edge of weight $\frac{1}{c_{ij}}$. The authors then compute wish-distances
$\delta_{ij}$ as the shortest path between node $i$ and $j$ in the graph.
Constructing wish-distances in such a way has two advantages: (i) low contact
counts do not contribute much to the wish-distances; (ii) the resulting
wish-distances form a distance matrix, and thus an optimal solution to the MDS
can be found.

In addition of having subtle differences in the derivation of wish distances,
the different methods also vary by the inclusion of weights to reflect
confidence in ``wish-distances'' \citep{ay:three-dimensional}, or constraints
to reflect prior knowledge on the structure. Table~\ref{table:mds_detail}
summarizes the characteristic of each method.

\begin{table*}
\scriptsize
\centering
\begin{tabular}{rccccccccc}
\hline
\multirow{2}{*}{\textbf{Paper}} & \multirow{2}{*}{\textbf{Organism}} &
\textbf{Full or}
& \textbf{Counts-to-distance} &
\multicolumn{4}{c}{\multirow{2}{*}{\textbf{Constraints}}}
& \multirow{2}{*}{\textbf{Weights}}\\
 & & \textbf{partial} & \textbf{mapping} &  \\
 \cmidrule(lr){5-8} 
 & & & & Adj. & rDNA & C & T & \\
\hline
\cite{dekker:capturing} & {\em S. cerevisiae} & Chr III & Worm-like chain
behavior & & & & \\
\cite{duan:three-dimensional} & {\em S. cerevisiae} & Whole-genome & Linear
relationship & \CHECK &  \CHECK & \CHECK &  & \\
\cite{tanizawa:mapping} & {\em S. pombe} & Whole-genome &
FISH-derived relationship & \CHECK & \CHECK & \CHECK & \CHECK &  \\
\cite{ay:three-dimensional} & {\em P. falciparum} & Whole-genome & Fractal
globule derived relationship & \CHECK  & & &
 & $\frac{1}{\delta_{ij}^2}$ \\
\cite{lesne:3d} & & Whole-genome & \textit{ad hoc} & & & & & \
\end{tabular}
\caption{Differences between MDS-based methods}{}
\label{table:mds_detail}
\end{table*}


\subsubsection*{Non-metric MDS-based method}

A crucial step of MDS-based method is the conversion of counts into
wish-distances. As described earlier, converting contact counts into
wish-distances require strong assumptions that may not be met in practice. For
example, this mapping changes from one organism to another
\citep{fudenberg:higher-order}, from one resolution to another
\citep{zhang:inference} or even from one time point to another during the cell
cycle \citep{le:high-resolution, ay:three-dimensional}. Several methods have
since been proposed to alleviate this problem.

\citet{ben-elazar:spatial} casts the inference as a \emph{non-metric MDS}
\citep{kruskal:multidimensional}, where the 3D structure is inferred jointly
with the wish-distances. The authors first filter the
interaction counts to keep only the most significant interactions and
interpolate the missing values to obtain a smooth, symmetric positive definite
matrix.

\citet{zhang:inference} proposes to parametrize the count-to-distance mapping
as a power-law ($d = \beta c^\alpha)$, and to infer jointly the parameters
jointly with the structure. In addition, the authors propose to add a
penalization term for non-interacting beads. The proposed penalization allows
to cast the problem as a convex optimization and thus guarantees to retrieve
the optimal structure.

\subsubsection*{Other ad-hoc optimization methods}

A series of methods cast the inference as an optimization method, where the
objective function is designed by hand to fulfill a number of properties
\citep{trieu:large, trieu:MOGEN, trieu:3D}. For example, \citep{trieu:large}
formulates an optimization problem that attempts to place the beads such that,
if a significant contact is observed between $i$ and $j$, then $\| x_i - x_j
\| < d_c$. On the other hand, if a no significant contact is observed, then
the two beads should be no closer than $d_c$. In addition, the authors also
place constraints on the minimum distance separating two beads and maximum
distances between two beads. To satisfy all those properties, the authors
propose to minimize the following objective function:

\begin{equation*}
\begin{aligned}
\mathcal{O}(X, \delta_{ij}) = & \underset{i, j | c_{ij} \neq 0, |i-j| \neq 1}{\sum} \Bigg(W_1 \text{tanh}(d^2_c -
d^2_{ij}) c_{ij} + W_2 \frac{\text{tanh}(d^2_{ij} - d^2_\text{min})}{N} \Bigg) \\
& \underset{i, j | c_{ij} = 0}{\sum} \Bigg(W_3 \frac{\text{tanh}(d^2_\text{max} -
d^2_{ij})}{N} + W_4 \frac{\text{tanh}(d^2_{ij} -
d^2_{c})}{N}\Bigg) \\
& \underset{i, j | |i - j| = 1}{\sum} \Bigg(W_1 \frac{\text{tanh}(d^2_\text{amax} -
d^2_{ij})}{N} + W_2 \frac{\text{tanh}(d^2_{ij} -
d^2_\text{min})}{N}\Bigg) \,,\\
\end{aligned}
\end{equation*}

where $N$ is the total number of interaction, $(W_1, W_2, W_3, W_4)$ weights
associated to each term, $d_\text{amax}$ the maximum distance separating two
adjacent beads, $d_\text{min}$ the minimum distance separating two adjacent
beads and $d_\text{max}$ the maximum distance between two beads. 
This method is then generalized to include \textit{trans} contact counts data
\citep{trieu:MOGEN}

To give another example of carefully crafted optimization, \cite{trieu:3D}
proposes to exploit the Lozentzian function, to formulate a constraint-based
problem. They first convert the contact counts $c_{ij}$ into wish-distances,
and cast the following optimization problem:

\begin{equation*}
\renewcommand{\arraystretch}{2}
\begin{array}{ccl}
\underset{\Xb}{\text{maximize}} & & \underset{| i - j | = 1}{\sum} \frac{c^2
c_\text{max}}{c^2 + (d_{ij} - \delta_{ij})} + 
\underset{| i - j | \neq 1}{\sum} \frac{c^2 c_{ij}}{c^2 + (d_{ij} - \delta_{ij})}
\end{array}
\end{equation*}

This objective problem attempts to minimize the difference between the
wish-distance $\delta_{ij}$ and the euclidean distance $d_{ij}$ between bead
$i$ and $j$ similarly to MDS-based method, but exploiting the property of the
Loretzian function to have very high derivative when constraints are broken.


\subsubsection*{Statistical-based models}

\citet{varoquaux:statistical} proposes a novel approach called \texttt{Pastis}
based on statistical modeling of contact counts as random Poisson variables
where the intensity is a function of the distance: $c_{ij} \sim
\text{Poisson}(\beta d_{ij}^\alpha)$, and some perform the optimization only
including a subset of contact counts. The authors propose to cast the
inference of 3D models as maximizing the likelihood:

\begin{equation}
\renewcommand{\arraystretch}{2}
\begin{array}{cll}
\underset{\alpha, \beta, \textbf{X}}{\text{max}} &
\mathcal{L}(\mathbf{X}, \alpha, \beta) = \underset{i<j\leq n}{\sum}  c_{ij}
\alpha \log d_{ij} + c_{ij} \log \beta - \beta d_{ij}^\alpha &\\
\end{array}
\end{equation}


\texttt{Pastis} can automatically adjust the parameters $\beta$, $\alpha$ of
the counts-to-distance transfer function and infer a genome structure that
best explains the observed data.


\subsection*{Ensemble methods}

{\em Ensemble approaches} aim at inferring a population of structures
representative of the contact count map. The methods fall into two distinct
categories: the first type of problem casts a non-convex or ill-defined
optimization problem and sample local minima of the function
\citep{bau:three-dimensional, umbarger:three-dimensional}, while the second
type proposes a statistical modeling of the problem and samples the posterior
distribution \citep{rousseau:three, hu:bayesian}.A parallel can be made
between the first group of methods and {\em Consensus} MDS type of approaches,
and the second with {\em Consensus} statistical based methods.

\subsubsection*{Sampling local minima}

\citet{umbarger:three-dimensional,bau:three-dimensional, kalhor:genome} models
chromosomes as a series of beads, linked by restraining oscillators. These
oscillators can be thought of as a ``force'' between beads so that they come
into contact or ensure a minimal or maximal distance between those. The models
includes three differente types of restraints: (i) beads seen interacting are
restrained with harmonic oscillators of strengths derived from the contact
counts; (ii) adjacent beads are ensured to be neither too close nor too far
from one another; (iii) the last oscillators are activated only when
constraints are not fulfilled to XXX.  This yields an optimization problem
with a large number of local minima, which the authors sample from by running
50,000 minimizations starting from random starts.

\subsubsection*{Estimating the posterior distribution}

Similarly to \citet{varoquaux:statistical}, \citet{rousseau:three} and
\citet{hu:bayesian} proposes to model contact counts with a formal
probabilistic model. \citet{rousseau:three} models observed contact counts
$c_{ij}$ as a random Gaussian variable of mean $\beta d_{ij}^{\alpha}$,
$\alpha \leq 0$ and variance $\sigma_{ij}$ estimated directly from the contact
count data, while \citet{hu:bayesian} models contact counts as random Poisson
variables of mean $\beta d_{ij}^\alpha$. The authors then sample from the
posterior using MCMC.

\subsection*{Single-cell models}

The last category of data-driven methods to infer the 3D architecture of the
genome rely on a new protocol to probe single-cell for their 3D structures
\citep{nagano:single-cell,ramani:massively}. Single-cell Hi-C is still in its early days,
and only a handful of data sets are today publicly available. The contact maps
originating from these data sets are very sparse, and specific methods to
infer 3D structures need to be developed specifically for sc-HiC. The
challenge of constructing models from this data has been tackled very
differently by two groups. 

Akin to {\em consensus} MDS-like method and {\em ensemble}-local minima
methods,  the first approach is to consider each contact as a constraint and
to formulate
an under constrained optimization problem. A population of structures
satistifying the constraints can be found by sampling local minimas.

\citet{paulsen:manifold} proposes a very different method, and formulate a
\textit{manifold based optimization}, where a low rank PSD matrix (and thus a
distance matrix) is optimized to be as close as possible to the sparse contact
count matrix. Applying classical MDS on this low rank PSD matrix then yields a
3D model of the genome.

\begin{table}[ht!]
\begin{center}
\footnotesize
\begin{tabular}{rlccccc}
\hline
\multirow{2}{*}{\textbf{\footnotesize Publication}} & \multirow{2}{*}{\textbf{\footnotesize Name}} &
\textbf{\footnotesize Consensus or} 
&\multirow{2}{*}{\textbf{\footnotesize MDS-based}} & \textbf{\footnotesize Statistical} &
\multirow{2}{*}{\textbf{\footnotesize Available}} \\
& & \textbf{\footnotesize Ensemble}
 & & \textbf{\footnotesize model} & \\
\hline
\hline
\footnotesize{\cite{dekker:capturing}} & & C & & & \\
\footnotesize{\cite{duan:three-dimensional}} &  & C & \CHECK & & \CHECK \\
\footnotesize{\cite{tanizawa:mapping}} & & C & \CHECK & & \\
\footnotesize{\cite{ay:three-dimensional}} & & C & \CHECK & & \\
\footnotesize{\cite{ben-elazar:spatial}} & & C & \CHECK & & \CHECK \\
\footnotesize{\cite{varoquaux:statistical}} & Pastis & C & & \CHECK & \CHECK\\
\footnotesize{\cite{bau:three-dimensional}} & & E & & &  \\
\footnotesize{\cite{umbarger:three-dimensional}} & & E  & & &\\
\footnotesize{\cite{zhang:inference}} & chromSDE & C &  \CHECK & & \CHECK\\
\footnotesize{\cite{rousseau:three}} & & E & & \CHECK & \CHECK\\
\footnotesize{\cite{hu:bayesian}} & Bach & E/C &  & \CHECK & \CHECK\\
\footnotesize{\cite{kalhor:genome}} & & E &   & &\\
\footnotesize{\cite{lesne:3d}} & ShRec3D & C  & \CHECK & & \CHECK \\
\footnotesize{\cite{trieu:large}} & & C & & & \\
\footnotesize{\cite{trieu:3D}} & & C & &  & \\
\footnotesize{\cite{trieu:MOGEN}} & MOGEN & C & & & \\
\footnotesize{\cite{nagano:single-cell}} & & E & & & \\
\footnotesize{\cite{paulsen:manifold}} & & C & \CHECK & & \CHECK \\
\hline
\end{tabular}
\end{center}
\caption{\bf A comparison of 3D inference methods}{In this table, we summarize
properties of published methods to infer the 3D structure of the genome: (1)
is it a consensus or a ensemble based inference? (2) Is it an MDS based
method? (3) or relies on a statistical modeling; (4) is the software
available or not (to the best of our knowledge).}
\end{table}

\section*{Modeling DNA as a flexible fibers under constraints}

In the previous section, we discussed how from a contact count map one could
infer a {\em consensus} model or a population of models that best represented
the contact map. Another set of methods approach the problem of finding 3D
models of genome architecture very differently. Instead of using data-driven
approaches, these methods model chromosomes as series of beads or flexible
fibers, and attempt to find the smallest set of constraints such that a
generated contact maps from these models match as closely as possible the
observed contact maps. As these models refine the constraints, more insight on
the hallmarks of the genome architecture are discovered.


\subsection*{A small set of constraints explain the 3D architecture of \textit{S. cerevisiae}}

The budding yeast \textit{S. cerivisiae}'s 3D structure has been extensively
studied, both through 3C-type studies \citep{dekker:capturing,
duan:three-dimensional, burton:} and through bio-imaging experiments
\citep{XX}. The small size of its genome, the well-known hallmarks of its
genome architecture and the availability of high resolution contact maps and
FISH data set quickly led several teams to investigate the minimal set of
constraints needed to reproduce the hallmarks of its genome architecture.

\cite{tjong:physical, tokuda:dynamical, tjong:physical} model {\em S.
cerevisiae}'s chromosomes as flexible random fibers under a small set of
constraints. While the exact modeling proposed by the three groups differ, the
set of constraints are similar: \textit{(i)} the chromosomes are constrained
into a spherical ball representing the nucleus; \textit{(ii)} centromeres are
constrained into a spherical ball tethered to the nuclear membrane;
\textit{(iii)} telomeres are tethered to the nuclear membraine; \textit{(iv)}
rDNA is constrained into the nucleolus, represented as a spherical ball
opposite to the centromeres.

Let us take a closer look at the model proposed by \citet{tjong:physical}.
The 16 chromosomes are modeled as beads-on-a-string spaced 3.2kb apart. The
authors cast an optimization problem including the following restraints: 

\begin{itemize}
\item \textbf{Adjacent Beads Constraints} Two adjacent beads spaced 3.2kb
apart are constrained to be 30~nm apart;
\item \textbf{Volume Exclusion Constraints} Each pairs of beads are spaced at
minimum $30~nm$ apart;
\item \textbf{Nucleus Constraint} All beads lie in a spherical nucleus of
radius $1\mu m$
\item \textbf{Centromeres Constraints} Centromeres are localized in a small sphere of
50nm abutting the nuclear envelop.
\item \textbf{Telomere Constraints} Telomeres are located at most 50nm from the nuclear envelop.
\item \textbf{rDNA restraint} The repeated rDNA located on chromosome 12 is
constrained to be inside the nucleolus;
\item \textbf{Nucleolus exclusion restraint} All beads that isn't rDNA is
constrained outside of the nucleolus;
\item \textbf{Chain stiffness}: an angular restraint is imposed on three
consecutive beads to reproduce chain stiffness:
\begin{equation}
\frac{1}{2} k_{\text{angle}} \sum^{N - 2}_{i = 1} \left( 1 - \frac{x_{i + 1} -
x_i}{\|x_{i + 1} - x_i\|} \cdot \frac{x_{i + 2} - x_{i + 1}}{\|x_{i + 2} -
  x_{i + 1}\|} \right)^2\,,
\end{equation}
\end{itemize}

In the end, the objective function minimized is the following:

\begin{equation}
\end{equation}

Using IMP, the authors generate a 25,000 structures, all of which respect the
constraints listed above. The population of structures is then used to created
a ``volume-exclusion contact map'', considering that two beads that are less
then 45~nm apart from a contact. The Pearson correlation of the
volume-exclusion contact map and the Hi-C one are highly correlated,
demonstrating this small set of constraints fully explain the observed counts.
In addition, the population of structures also explain FISH experiments
previously published.


\subsection*{\textit{P. falciparum}'s 3D structure cannot be explained by a
constrained volume exclusion model}


Running a similar model on {\em P. falciparum}

\section*{Downstream analysis using 3D models: a highlight of the study of
{\em P. falciparum}'s 3D structure}

\subsection*{Profiling {\em P. falciparum}'s genome architecture during the
asexual cycle}
\label{sec:data}

In the past year, two groups have published novel dataset studying {\em P.
falciparum}'s 3D genome architecture using Hi-C. \citet{lemieux:genome-wide}
studied several strains associated with populations expression unique
\textit{var} genes, in order to study key folding properties of the 3D
structure relating to \textit{var} gene expression. On the other hand,
\citet{ay:three-dimensional} focus three key time-point of the development
cycle. Table~\ref{tab:data} summarizes the available data sets and their key
properties after processing the data as described in
\citet{ay:three-dimensional} and combining all available libraries for each
life-cycle stage and strain.

\subsection*{Structure stability}

Both the constrained and unconstained problem stated above are not convex. The
reader may well ask how sensitive the result is to initialize. In short, it is
not. To study the stability of the results with respect to the initialization,
I perform for each stage presented here 1000 optimization, starting from
different initialization. 

The first question that arise is simply: are structures from the same time
points but from a different start ponit more alike than structures from
different time points? To answer this question, I compute for each structure
and each time point a feature matrix, composed of a subsampled pairwise
euclidean distances of each structure (so that the resulting feature matrix
fits in memory on a laptop). I then compute a stochastic PCA on this feature
matrix and plot the projection in 2D space using the 3 first components. The
results clearly show that structures from the same time point cluster
strongly. Repeating this experiment on singly chromosomes yields the same
result.

The second question one can ask is how do the structures differ. Tackling this
question is very challenging, but can be reformulated in a much easier way.
Are the hallmarks of interest conserved across structures of a same stage?
\citet{ay:three-dimensional} and \citet{lemieux:genome-wide} both identified
{\em P. falciparum} folded in very specific ways, with VRSSM genes highly
interacting. \citet{ay:three-dimensional} also observed strong clustering of
the centromeres, and enrichment in interaction at the telomeres. These
observation can lead to a rigourous approach at identifying whether families
of loci clustered in the structures and even quantifying this in a rigourous
manner.

\subsection*{3D gene set enrichment}

To assess whether groups of genes are colocolized in a 3D model,
\citet{ay:three-dimensional} leveraged a statistical method developped by
\citet{witten:assessment}, which requires labeling each pair of loci in two groups:
"close" or "far". 

\fixme{This is unclear. Introduce notatinos and rewrite this section}

The authors used varying distance
thresholds (10\%, 20\% and 40\% of the nuclear diameter) to deem a locus pair
“close” and labeled all remaining
pairs in the set as “far”.  The authors then compare the enrichment of loci
pairs of a group being close and far by resampling loci amongst a same
chromosome.

This approach dichotomizes loci pairs into two groups, and check for the
enrichment of a label in one of the two groups. \citep{capurso:distance-based}
presents an approach that avoids this step, and instead directly estimates the
significance within the 3D model. Briefly, for a group $\mathcal{G}$, MPED
computes a test statistics:
\begin{equation*}
M = \underset{i,j \in \mathcal{G}| c_i != c_j}{\text{median}} d_{ij}\,,
\end{equation*}

where $d_{ij}$ is the euclidean distance between bead $i$ and bead $j$. The
null distribution is estimated empirically by resampling $10^5$ with
preservation of the chromosome structure. If the $M$ statistics is smaller
than the mean of the null distribution, it is compared to the lower tail of
the distribution and indicates co-localization. If the $M$ statistics is
larger than the mean of the null distribution, then it is compared to the
upper tail of the distribution, and indicates dispersion.

Applying MEPD to the Ring stage, \citet{capurso:distance-based} confirm that
centromers, telomers, VRSM genes (both overall, subtelomeric and internal)
colocalize.


\begin{table}
\scriptsize
\centering
\begin{tabular}{cccccccc}
\hline
\textbf{Name} & \textbf{Strain} & \textbf{Stage} & \textbf{Resolution} &
\textbf{Number of contacts} & \textbf{Perc. of \textit{cis}} & \textbf{Perc of
\textit{trans}}& \textbf{Reference} \\
\hline
\hline
Ay-\textit{rings} & 3D7 & Late Rings & 10~kb & 16711552 & 43\% & 57\% &  \cite{ay:three-dimensional} \\
Ay-\textit{trophozoites} & 3D7 & Trophozoites &10~kb & 56348498 & 53\% & 47\% & \cite{ay:three-dimensional} \\
Ay-\textit{schizonts} & 3D7 & Schizonts & 10~kb & 11652832 & 55\% & 45 \% & \cite{ay:three-dimensional} \\
Lemieux-\textit{A4+} & IT/BC6+ & Rings & 25~kb & 18488252 & 19\% & 81\% & \cite{lemieux:genome-wide} \\
Lemieux-\textit{A4} & IT/3G8 & Rings &  25~kb & 19674672 & 28\% & 72\% & \cite{lemieux:genome-wide}\\
Lemieux-\textit{A44} & IT/BC6- & Rings & 25~kb & 18660594 & 25\% & 75\% & \cite{lemieux:genome-wide}\\
Lemieux-\textit{DCJ\_On} & NF54/DCJ on & Rings & 25~kb & 3098370 & 26\% & 74\% &\cite{lemieux:genome-wide} \\
Lemieux-\textit{DCJ\_Off} & NF54/DCF Off & Rings & 25~kb & 2533470 & 26\% & 73\% &\cite{lemieux:genome-wide} \\
Lemieux-\textit{B15C2} & NF54/B15C2 & Rings & 25~kb &  1022996 & 12\% & 88\% & \cite{lemieux:genome-wide}\\
\hline
\end{tabular}
\caption{Summary of the available {\em P. falciparum} Hi-C datasets}
\end{table}

\subsection*{Integrative analysis of gene expression and 3D structure using KernelCCA}



\section*{Acknowledgments}

{\it Conflict of Interest}: None declared.


\bibliographystyle{DeGruyter}
\bibliography{refs}


\begin{figure}[!p]
\centering
%\includegraphics{fig1}
\caption{}
\label{Fig1}
\end{figure}

\end{document}
